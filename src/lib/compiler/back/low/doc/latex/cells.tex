\section{Registers}

MLRISC uses
the \mlrischref{src/lib/compiler/back/low/code/registerkinds.api}{Registerkinds} 
interface to define all readable/writable resources
in a machine architecture,  or \emph{registers} 
The types defined herein are:
\begin{itemize}
 \item \sml{registerkind} -- different ilks of registers are assigned
   difference registerkinds.  The following registerkinds should be present
   \begin{itemize}
     \item \sml{GP} -- general purpose registers.
     \item \sml{FP} -- floating point registers.
     \item \sml{CC} -- condition code registers.
   \end{itemize}
   In addition, the registerkinds \sml{MEM} and \sml{CTRL}
   should also be defined.  These are used for representing
   memory based data dependence and control dependence.
   \begin{itemize}
     \item \sml{MEM} -- memory 
     \item \sml{CTRL} -- control dependence
   \end{itemize} 
 \item \sml{regmap} -- \href{regmap.html}{register map}
 \item \sml{registerset} -- a registerset represent a set of registers.  This
   type can be used to denote live-in/live-out information.  Registersets are
   implemented as immutable abstract types.
\end{itemize}

These core definitions are defined in the following api
\begin{SML}
api \mlrischref{src/lib/compiler/back/low/code/registerkinds-junk.api}{Registerkinds_Junk} =
sig
   eqtype registerkind 
   type register = int
   type regmap = register int_map.intmap
   exception REGISTERS

   my registerkinds : registerkind list 
   my registerkindToString : registerkind -> String
   my firstPseudo : register                    
   my Reg   : registerkind -> int -> register
   my GPReg : int -> register 
   my FPReg : int -> register
   my registerRange : registerkind -> {low:int, high:int}
   my make_register   : registerkind -> 'a -> register 
   my registerKind : register -> registerkind         
   my updateRegisterKind : register * registerkind -> Void        
   my numRegister   : registerkind -> Void -> int              
   my maxRegister   : Void -> register
   my make_reg    : 'a -> register              
   my new_freg   : 'a -> register              
   my newVar    : register -> register
   my regmap    : Void -> regmap
   my lookup    : regmap -> register -> register
   my reset     : Void -> Void
end
\end{SML}

\begin{itemize}
  \item\sml{registerkinds} -- this is a list of all the registerkinds defined in the
architecture
  \item\sml{registerkindToString} -- this function maps a registerkind into its name
  \item\sml{firstPseudo} -- MLRISC numbered physical resources
   in the architecture from 0 to firstPseudo-1.  
   This is the first usable virtual register number.
  \item\sml{Reg} -- This function maps the $i$th physical
   resource of a particular registerkind to its internal encoding used by MLRISC.
   Note that all resources in MLRISC are named uniquely.
  \item\sml{GPReg} -- abbreviation for \sml{Reg GP} 
  \item\sml{FPReg} -- abbreviation for \sml{Reg FP} 
  \item \sml{registerRange} -- this returns a range \sml{{low, high}}
   when given a registerkind, with denotes the range of physical resources
  \item \sml{make_register}  -- This function returns a new virtual register 
   of a particular registerkind.
  \item \sml{make_reg} -- abbreviation as \sml{make_register GP}
  \item \sml{new_freg} -- abbreviation as \sml{make_register FP}
  \item \sml{registerKind}  -- When given a register number, this returns its
    registerkind.  Note that this feature is not enabled by default.
  \item \sml{updateRegisterKind} -- updates the registerkind of a register.
  \item \sml{numRegister} -- returns the number of virtual registers allocated for one registerkind.
  \item \sml{maxRegister} --  returns the next virtual register id.
  \item \sml{newVar}  -- given a register id, return a new register id of
     the same registerkind.
  \item \sml{regmap} -- This function returns a new empty regmap
  \item \sml{lookup} -- This converts a regmap into a lookup function.
  \item \sml{reset} -- This function resets all counters associated
with all virtual registers.
\end{itemize}

\begin{SML}
api Registers = sig
   include api Registerkinds_Junk
   my GP   : registerkind 
   my FP   : registerkind
   my CC   : registerkind 
   my MEM  : registerkind 
   my CTRL : registerkind 
   my to_string : registerkind -> register -> String
   my stackptrR : register 
   my asmTmpR : register  
   my fasmTmp : register 
   my zeroReg : registerkind -> register option

   type registerset

   my empty      : registerset
   my add_register    : registerkind -> register * registerset -> registerset
   my drop_register    : registerkind -> register * registerset -> registerset
   my add_reg     : register * registerset -> registerset
   my drop_reg     : register * registerset -> registerset
   my add_freg    : register * registerset -> registerset
   my drop_freg    : register * registerset -> registerset
   my get_register    : registerkind -> registerset -> register list
   my update_register : registerkind -> registerset * register list -> registerset

   my registersetToString : registerset -> String
   my registersetToString' : (register -> register) -> registerset -> String

   my registersetToRegisters : registerset -> register list
end
\end{SML}

\begin{itemize} 
  \item \sml{to_string} -- convert a register id of a certain registerkind into
its assembly name.
  \item \sml{stackptrR} -- the register id of the stack pointer register. 
  \item \sml{asmTmpR} -- the register id of the assembly temporary 
  \item \sml{fasmTmp} -- the register id of the floating point temporary
  \item \sml{zeroReg} -- given the registerkind, returns the register id of the
   source that always hold the value of zero, if there is any. 
  \item \sml{empty} -- an empty registerset
  \item \sml{addRegister} -- inserts a register into a registerset
  \item \sml{rmvRegister} -- remove a register from a registerset
  \item \sml{addReg} -- abbreviation for \sml{addRegister GP}
  \item \sml{rmvReg} -- abbreviation for \sml{rmvRegister GP} 
  \item \sml{addFreg} -- abbreviation for \sml{addRegister FP}
  \item \sml{rmvFreg} -- abbreviation for \sml{rmvRegister FP} 
  \item \sml{allocate_register} -- look up all registers of a particular registerkind from
the registerset
  \item \sml{updateRegister} -- replace all registers of a particular registerkind
from the registerset. 
   \item \sml{registersetToString} -- pretty print a registerset 
   \item \sml{registersetToString'} -- pretty print a registerset, but first
apply a regmap function.
   \item \sml{registersetToRegisters} -- convert a registerset into list form.
\end{itemize}
