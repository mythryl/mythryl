
Mythryl's {\tt Null\_Or} facility corresponds closely to the {\tt C/C++} {\sc NULL} pointer 
facility.  The critical difference is that {\tt C/C++} programs can and do crash at 
runtime when the code attempts to dereference a {\sc NULL} pointer, whereas in Mythryl 
static compile-time checking guarantees that this can never happen.

A type {\tt Foo} is converted to one which also allows {\sc NULL} values via the 
{\tt Null\_Or} type function:  {\tt Null\_Or( Foo )}.

Mythryl type-checking then requires that values of this type always be checked 
for {\sc NULL} values before being used in a computation, typically via code like

\begin{verbatim}
    fun print_string  (maybe_string: Null_Or( String ))
	=
	case (maybe_string)
	NULL => ();
	string => printf ``Your string is '%s'.'' string;
	esac;
\end{verbatim}
