\documentclass{book}
%
%     A test doc so I can experiment without
%     having to regenerate the entire book document.
%
\usepackage{url}
\usepackage{fullpage}
\usepackage{isolatin1}
\usepackage{html}
\usepackage{hevea}
% \usepackage{makeidx}   % See http://hevea.inria.fr/doc/manual033.html#toc83
\usepackage{index}       % See http://hevea.inria.fr/doc/manual-packages.html#multind
\usepackage[english]{babel}

\newindex{api}{idx}{ignoredbyhevea}{Api Index}
\newindex{pkg}{idx}{ignoredbyhevea}{Package Index}

% Hevea interaction with .css style sheets is documented in
%     http://hevea.inria.fr/doc/manual019.html
\loadcssfile{local.css}
%
\def\homedir{http://mythry.org/latest}
\begin{latexonly}
\gdef\url#1#2{#2}
\gdef\ahrefurl#1{\url{#1}{{\tt #1}}}
\end{latexonly}

% Text specified in htmlhead and htmlfoot will be
% replicated by hacha in every .html file generated.
% For additional info see http://hevea.inria.fr/doc/cutname.html#toc19
% \htmlhead{header}
\htmlfoot{
\begin{rawhtml}
<HR SIZE=2>
\end{rawhtml}
\begin{center}
{\tiny Comments and suggestions to: \mailto{bugs@mythryl.org}}
\end{center}
}
%
% By default (for books) Hacha generates one html file per chapter.
% That is way too coarse-grain for us, so we change it to one per section.
% This is documented at http://hevea.inria.fr/doc/cutname.html#toc20
\renewcommand{\cuttingunit}{section}

%
% By default Hacha includes one level below the ``cutting unit'' in
% the table of contents.  Since we set the cutting unit finer than
% usual, but do not want the table of contents blowing up, we compensate.
% This stuff is documented in   http://hevea.inria.fr/doc/cutname.html#toc20
\setcounter{cuttingdepth}{0}	% default value is 1.
%
% NB: book.css gets generated by hacka -- see http://hevea.inria.fr/doc/cutname.html#toc19


% \makeindex


\title{Why Mythryl?}
\date{}
\author{}
\pagestyle{empty}
\begin{document}
\maketitle
\thispagestyle{empty}

\urldef{\smlnj}{\url}{http://www.smlnj.org/}
\urldef{\perl}{\url}{http://www.perl.org/}
\urldef{\texmacs}{\url}{http://www.texmacs.org/}
\urldef{\pronto}{\url}{http://www.muhri.net/pronto/}
\urldef{\mailman}{\url}{http://www.gnu.org/software/mailman/mailman.html}
\urldef{\exim}{\url}{http://www.exim.org/}
\urldef{\xbuffy}{\url}{http://www.fiction.net/blong/programs/xbuffy/}
\urldef{\avs}{\url}{http://www.avs.com/index_nf.html}
\urldef{\geomview}{\url}{http://www.geomview.org/}
\urldef{\blender}{\url}{http://www.blender.org/}
\urldef{\mlud}{\url}{http://www.moonflare.com/code/mlud/summary.php}
\urldef{\nyquist}{\url}{http://www.cs.cmu.edu/~music/nyquist/}
\urldef{\ardour}{\url}{http://ardour.org/}
\urldef{\cinelerra}{\url}{http://cvs.cinelerra.org/about.php}
\urldef{\graphviz}{\url}{http://www.graphviz.org/}


~~~~~~~~~~\verb|code|\verb*|1 2  3    4                                                                                         |\verb|#|\newline
~~~~~~~~~~\verb|code|\verb|1234567812345678123456789123456781234567812345678912345678123456781234567891234567812345678123456789|\verb|#|\newline
\hspace*{5.5815em}\verb|code|\hspace*{116.25em}\verb|#|\newline


\begin{quote}\begin{tiny}
                 ``There is no excellent beauty\newline
                 ~~that hath not some strangeness\newline
                 ~~in the proportion.''\newline
                 ~~~~~~~~~~~~~~~~~~~~~~~~~~~~---{\em Francis Bacon}

\end{tiny}\end{quote}


\begin{description}
\item[Engineered.] Mythryl is not just a bag of features like C++ or Python;  It has a design with provably good properties.
\item[Modern type system.] Stronger, more flexible typring than C++ or Java, yet rarely an explicit type declaration:  Hindley-Milner typechecking is a quantum jump beyond.
\item[Generic by design]  These days everyone is kludging in generics as an aftermarket feature.  Mythryl generics were designed in from day one.
\item[Typesafe.]  Never a \verb#.core# dump.
\item[Great namespace management.]  No more long ugly function names because everything C is a static or a global.
\item[Less coding effort.]  A variety of sources report productivity gains of $2\times$ to $10\times$.
\item[Minimal side-effects.]  Pre-adapted for the multi-core era, in which every side effect is a bug waiting to happen.
\item[Fast.]  Designed from day one for efficient compilation;  implemented via an incremental compiler capable of compiling individual statements to native code, in-memory.
\item[Great garbage collection.]  Serious state of the art multi-generation garbage collection, not wimpy mark-and-sweep.
\item[High-performance multiprogramming.]  Stackless implementation makes thread-switching a hundred times faster than contemporary languages.
\item[Fun!]  Mythryl puts the magic back.
\end{description}

\begin{rawhtml}
<center>
<table align="center"><tr>
"There is no excellent beauty<br>
that hath not some strangeness<br>
in the proportion."'<br>
<i>Francis Bacon</i>
</table>
</center>
\end{rawhtml}


\begin{description}
\item[Engineered.] Mythryl is not just a bag of features like C++ or Python;  It has a design with provably good properties.
\item[Modern type system.] Stronger, more flexible typring than C++ or Java, yet rarely an explicit type declaration:  Hindley-Milner typechecking is a quantum jump beyond.
\item[Generic by design]  These days everyone is kludging in generics as an aftermarket feature.  Mythryl generics were designed in from day one.
\item[Typesafe.]  Never a \verb#.core# dump.
\item[Great namespace management.]  No more long ugly function names because everything C is a static or a global.
\item[Less coding effort.]  A variety of sources report productivity gains of $2\times$ to $10\times$.
\item[Minimal side-effects.]  Pre-adapted for the multi-core era, in which every side effect is a bug waiting to happen.
\item[Fast.]  Designed from day one for efficient compilation;  implemented via an incremental compiler capable of compiling individual statements to native code, in-memory.
\item[Great garbage collection.]  Serious state of the art multi-generation garbage collection, not wimpy mark-and-sweep.
\item[High-performance multiprogramming.]  Stackless implementation makes thread-switching a hundred times faster than contemporary languages.
\item[Fun!]  Mythryl puts the magic back.
\end{description}


$\{ i^{2} | 1 < i < 100, i \epsilon Primes \}$


\begin{rawhtml}
<center><img src="cynbe1-web.jpg"></center>
\end{rawhtml}

\begin{center}
``{\em Putting the} {\tt fun} {\em back in hacking!}''
\end{center}




\end{document}

