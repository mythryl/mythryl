\chapter{Roadmap}

% ================================================================================
% This chapter is referenced in:
%
%     doc/tex/book.tex
%

% ================================================================================
\section{Overview}

\begin{quote}\begin{tiny}
           ``For I dipped into the future,\newline
            ~~~~~~~~~~far as human eye could see,\newline
            ~~Saw the Vision of the world,\newline
            ~~~~~~~~~~and all the wonder that would be;\newline
            ~~Saw the heavens fill with commerce,\newline
            ~~~~~~~~~~argosies of magic sails,\newline
            ~~Pilots of the purple twilight,\newline
            ~~~~~~~~~~dropping down with costly bales;''\newline

            ~~~~~~~~~~~~~~~~~~~~~~~~~---{\em Alfred, Lord Tennyson. 1842}
\end{tiny}\end{quote}

Mythryl's primary project goal is to provide a stable 
open-source production-quality {\sc POSIX}-flavored mostly-functional 
software development platform, developed by working programmers for 
working programmers.

To that end, changes to the core language and library functionality 
are in general made only when there exists prior art providing 
compelling evidence of substantial benefit to working programmers.

Developing such prior art is not the business of the Mythryl project, 
but rather of research projects such as \ahref{\smlnj}{\sc SML/NJ}: 
The cardinal virtues of production 
software platforms are not brilliance and innovation, but rather 
stability and predictability.  Working programmers want their compilers 
to just do the job.  Quietly, unobtrusively, and above all dependably.

(So don't be offended if we don't immediately incorporate your latest 
wiz-bang bleeding-edge patch!)

% ================================================================================
\section{Focal priorities}

\begin{quote}\begin{tiny}
        ``The best way to predict the future is to invent it.''\newline
            ~~~~~~~~~~~~~~~~~~~~~~~~~---{\em Alan Kay}
\end{tiny}\end{quote}


One current focus of the Mythryl project proper 
is improving stability and 
usability by fixing known bugs and improving documentation.

Another current focus is building and improving development-critical 
infrastructure, such as an {\sc IDE} and debugger, 
good interfaces to major C libraries, and an archive network for 
sharing, indexing and installing packages written in Mythryl.

% ================================================================================
\section{Mythryx}

A wider goal is to foster the development of a complete mostly-functional 
software ecology to eventually replace the current C-based open source 
software ecology, which is starting to smell distinctly ``past pull date''.

When it grows large enough, this effort will be split off as 
{\tt mythryx.org}, but for now work on this is folded into 
the main Mythryl distribution.

An initial prime focus of this effort is the development of 
a complete set of internet daemons written in Mythryl:

\begin{itemize}
\item A {\sc DNS} daemon.
\item An {\sc NTP} daemon.
\item An {\sc HTTP} daemon.
\item An {\sc SMTP} daemon.
\item An {\sc FTP} daemon.
\end{itemize}

These constitute low-hanging fruit because typesafe implementations 
of daemons directly exposed to the Internet are inherently more secure 
than daemons written in C, and attract significant interest and 
uptake on security grounds alone.

Careful coding to take advantage of the type system, as for example 
using types to maintain a firewall between untrusted strings of external 
origin and trusted strings of internal origin (in the spirit of 
\ahref{\perl}{\sc Perl}'s {\em taint} mechanism, but without the 
runtime penalty) can increase this architectural security advantage 
relative to C.
