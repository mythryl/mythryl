% --------------------------------------------------------------------------------
\subsection{Printf Sprintf Fprintf Overview}
\cutdef*{subsubsection}
\label{section:libref:printf-sprintf-fprintf:overview}

For convenience and to reduce the learning curve, Mythryl supports 
{\bf printf}, {\bf sprintf} and {\bf fprintf} functions patterned 
closely after those of C.

These are a bit of a hack in that their types appear to vary depending 
upon the format string.  This is implemented by a special-case tweak in 
the Mythryl parser which analyses the format string (which must be a 
string constant) and synthesizes an appropriate argument list. 

The core functionality is implemented in the \ahrefloc{pkg:sfprintf}{sfprintf} package 
from file \ahrefloc{src/lib/src/sfprintf.pkg}{src/lib/src/sfprintf.pkg}.  It may sometimes 
be useful to invoke that package directly, for example if the format string must be 
computed instead of a string constant.

\cutend*

% --------------------------------------------------------------------------------
\subsection{Printf Sprintf Fprintf Functions}
\cutdef*{subsubsection}
\label{section:libref:printf-sprintf-fprintf:functions}

The Mythryl {\bf printf}, {\bf sprintf}, and  {\bf fprintf} are curried, 
so argument follow without commas.  Argument expressions must be parenthesized. 
The {\bf printf} function prints to stdout, the {\bf sprintf} constructs and 
returns a string, and the {\bf fprintf} function writes to the specified stream; 
they are otherwise identical.


Example:
\begin{verbatim}
eval:  printf "%d\n" 12;
12

eval:  apply (printf "%d\n") [ 12, 13, 14 ];
12
13
14
\end{verbatim}

\cutend*

% --------------------------------------------------------------------------------
\subsection{Printf Sprintf Fprintf Format Strings}
\cutdef*{subsubsection}
\label{section:libref:printf-sprintf-fprintf:format-strings}

Mythryl supports the following basic format specifiers: 
\begin{itemize}
\item \%i Decimal integer.
\item \%o Octal integer.
\item \%x Hexadecimal integer, lowercase letters.
\item \%X Hexadecimal integer, uppercase letters.
\item \%b Binary integer.
\item \%e Float, exponential format, lowercase 'e'.
\item \%E Float, exponential format, uppercase 'e'.
\item \%f Float, fixed format.
\item \%G Float, general format, uppercase 'e'.
\item \%g Float, general format, lowercase 'e'.
\item \%c Character value.
\item \%B Boolean value.
\end{itemize}

Zero or more of following modifiers may follow the percent in a format specifier: 
\begin{itemize}
\item ' ' Print a leading blank on positive numbers.
\item '+' Print a leading plus on positive numbers.
\item '~' Print a leading tilda (instead of hyphen) on negative numbers.
\item '-' Left-justify (strings) or print negative numbers with a leading hyphen (numbers).
\item '#' Base.
\item '0' Zero-pad (instead of blank-padding).
\end{itemize}

A decimal field width is allowed after the above modifiers (if any).

In the case of the floating point specifiers (that is, 'E', 'e', 'f', 'G', 'g'), the 
width field may be followed by a decimal point and a decimal precision.

No other format specifiers or modifiers are currently supported.

Integer types other than Int must currently be converted to string form using other facilities, 
typically the {\bf to\_string} function in the relevant package or else the underlying 
\ahrefloc{pkg:sfprintf}{sfprintf} package.

(Mapping of basic format specifiers to meaning is done in 
\ahrefloc{src/lib/src/printf-field.pkg}{src/lib/src/printf-field.pkg}.)

Examples:
\begin{verbatim}
eval:  printf "%12s\n" "foo";
         foo

eval:  printf "%-12s\n" "foo";
foo         

eval:  printf "%12s%12s\n" "foo" "bar";
         foo         bar

eval:  printf "%-12s%-12s\n" "foo" "bar";
foo         bar         

eval:  printf "%B\n" ("foo"=="bar");
FALSE

eval:  printf "%f\n" pi;
3.141593

eval:  printf "%g\n" pi;
3.14159

eval:  printf "%e\n" pi;
3.141593e00

eval:  printf "%E\n" pi;
3.141593E00

eval:  printf "%15.7g\n" pi;
       3.141593

eval:  printf "%15.10g\n" pi;
    3.141592654

eval:  printf "%f\n" (pi*1000.0*1000.0);
3141592.653590

eval:  printf "%g\n" (pi*1000.0*1000.0);
3.14159e06

eval:  sprintf "%g\n" (pi*1000.0*1000.0);
"3.14159e06\n"
\end{verbatim}

\cutend*

% --------------------------------------------------------------------------------
\subsection{Printf Sprintf Fprintf  See Also}
\cutdef*{subsubsection}
\label{section:libref:printf-sprintf-fprintf:see-also}

See also:  \ahrefloc{pkg:sfprintf}{sfprintf} in \ahrefloc{src/lib/src/sfprintf.pkg}{src/lib/src/sfprintf.pkg}\newline 

\cutend*

